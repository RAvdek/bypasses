\documentclass[11pt]{amsart}
\usepackage[a4paper, margin=2.5cm]{geometry}
\usepackage[utf8]{inputenc}

%%%%Package Info%%%
\usepackage{overpic, amssymb, times, graphicx, tikz-cd}
\usepackage[all]{xy}
\usepackage[backref=page]{hyperref}
\usepackage{verbatim}
%%%%%%%%%%%%%%%%%%%%%%%%%%%%%%%

\hypersetup{
    colorlinks=true,
    linkcolor=blue,
    filecolor=blue,
    urlcolor=blue,
    citecolor=blue,
}
\urlstyle{same}

%%%%Specifying new commands%%%%%%%%%%
\DeclareMathOperator{\spec}{spec}
\DeclareMathOperator{\rank}{rank}
\DeclareMathOperator{\coker}{coker}
\DeclareMathOperator{\Span}{span}
\DeclareMathOperator{\sprod}{p}
\DeclareMathOperator{\cop}{cop}
\DeclareMathOperator{\im}{im}
\DeclareMathOperator{\dom}{dom}
\DeclareMathOperator{\Diff}{Diff}
\DeclareMathOperator{\Hom}{Hom}
\DeclareMathOperator{\End}{End}
\DeclareMathOperator{\ind}{ind}
\DeclareMathOperator{\sym}{sym}
\DeclareMathOperator{\Symp}{Symp}
\DeclareMathOperator{\Int}{int}
\DeclareMathOperator{\Ann}{Ann}
\DeclareMathOperator{\Id}{Id}
\DeclareMathOperator{\Maslov}{M}
\DeclareMathOperator{\CZ}{CZ}
\DeclareMathOperator{\bcs}{bcs}
\DeclareMathOperator{\Flow}{Flow}
\DeclareMathOperator{\ASL}{ASL}
\DeclareMathOperator{\SL}{SL}
\DeclareMathOperator{\GL}{GL}
\DeclareMathOperator{\PD}{PD}
\DeclareMathOperator{\tr}{tr}
\DeclareMathOperator{\rot}{rot}
\DeclareMathOperator{\lk}{lk}
\DeclareMathOperator{\tb}{tb}
\DeclareMathOperator{\wl}{wl}
\DeclareMathOperator{\cycword}{cw}
\DeclareMathOperator{\word}{w}
\DeclareMathOperator{\cross}{cr}
\DeclareMathOperator{\sgn}{sgn}
\DeclareMathOperator{\Diag}{Diag}
\DeclareMathOperator{\const}{const}
\DeclareMathOperator{\Ret}{Ret}
\DeclareMathOperator{\expdim}{expdim}
\DeclareMathOperator{\SFT}{SFT}
\DeclareMathOperator{\Sobalev}{W}
\DeclareMathOperator{\Ltwo}{L^{2}}

\newcommand{\thicc}[1]{\pmb{#1}}
\DeclareMathOperator{\aug}{\thicc{\epsilon}}

\newcommand{\Leg}{\Lambda}
\newcommand{\filtration}{\mathcal{F}}
\newcommand{\newRSFT}{PDH}
\newcommand{\Algebra}{\mathcal{A}}
\newcommand{\field}{\mathbb{F}}
\newcommand{\Proj}{\mathbb{CP}}
\newcommand{\C}{\mathbb{C}}
\newcommand{\R}{\mathbb{R}}
\newcommand{\Z}{\mathbb{Z}}
\newcommand{\N}{\mathbb{N}}
\newcommand{\Q}{\mathbb{Q}}
\newcommand{\disk}{\mathbb{D}}
\newcommand{\grad}{\nabla}
\newcommand{\bigO}{\mathcal{O}}
\newcommand{\region}{\mathcal{R}}
\newcommand{\Igrading}{\mathcal{I}_{\Lambda}}
\newcommand{\action}{\mathcal{A}}
\newcommand{\energy}{\mathcal{E}}
\newcommand{\rect}{\mathcal{D}}
\newcommand{\delbar}{\overline{\partial}}
\newcommand{\Cinfty}{\mathcal{C}^{\infty}}
\newcommand{\Rthree}{(\R^{3},\xi_{std})}
\newcommand{\Lie}{\mathcal{L}}
\newcommand{\Sthree}{(S^{3},\xi_{std})}
\newcommand{\Tthree}{(T^{3},\xi_{1})}
\newcommand{\Circle}{S^{1}}
\newcommand{\Lj}{L_{j}}
\newcommand{\lj}{\ell_{j}}
\newcommand{\Lp}{L^{+}}
\newcommand{\Lm}{L^{-}}
\newcommand{\Nde}{N^{(\delta,\,\epsilon)}}
\newcommand{\Njde}{N_{j}^{(\delta,\,\epsilon)}}
\newcommand{\Nab}{N^{(\alpha,\,\beta)}}
\newcommand{\Njab}{N_{j}^{(\alpha,\,\beta)}}
\newcommand{\Bthree}{(B^{3},\xi_{std})}
\newcommand{\LambdaZero}{\Lambda^{0}}
\newcommand{\LambdaPlus}{\Lambda^{+}}
\newcommand{\LambdaMinus}{\Lambda^{-}}
\newcommand{\LambdaPM}{\Lambda^{\pm}}
\newcommand{\SurgL}{M_{\Leg}}
\newcommand{\SurgXi}{\xi_{\Leg}}
\newcommand{\SurgLxi}{(\SurgL, \SurgXi)}
\newcommand{\SurgLxiClosed}{(\SurgLClosed, \SurgXi)}
\newcommand{\SurgLPrime}{\R^{3}_{\Lambda}}
\newcommand{\SurgXiPrime}{\xi_{\Lambda}}
\newcommand{\SurgLxiPrime}{(\SurgLPrime, \SurgXiPrime)}
\newcommand{\pxy}{\pi_{xy}}
\newcommand{\J}{\mathcal{J}}
\newcommand{\SLtwoR}{\SL(2, \R)}
\newcommand{\GLtwoR}{\GL(2, \R)}
\newcommand{\SymptwoR}{\Symp(2, \R)}
\newcommand{\half}{\frac{1}{2}}
\newcommand{\partialCH}{\partial_{CH}}
\newcommand{\partialLCH}{\partial_{LCH}}
\newcommand{\partialRSFT}{\partial_{RSFT}}
\newcommand{\partialSFT}{\partial_{SFT}}
\newcommand{\partialLRSFT}{\partial_{LRSFT}}
\newcommand{\ModSpace}{\mathcal{M}}

\newcommand{\be}{\begin{enumerate}}
\newcommand{\ee}{\end{enumerate}}
\newcommand{\s}{\vskip.1in}
\newcommand{\n}{\noindent}

\newcommand{\orbitVS}{V}
\newcommand{\SOT}{(S^{3},\xi_{OT})}
\newcommand{\Mxi}{(M,\xi)}
\newcommand{\MxiOT}{(M_{OT}, \xi_{OT})}
\newcommand{\MxiGamma}{(M, \xi, \Gamma)}
\newcommand{\AOB}{(\Sigma,\Phi)}
\newcommand{\OB}{(B,\Sigma,\Phi)}
\newcommand{\OBOT}{(S^{1}\times[0,1], D^{-}_{S^{1}\times\lbrace\frac{1}{2}\rbrace})}
\newcommand{\Xfancy}{\mathcal{X}}
\newcommand{\Cfancy}{\mathcal{C}}
\newcommand{\Gfancy}{\mathcal{G}}
\newcommand{\Hfancy}{\mathcal{H}}
\newcommand{\Sfancy}{\mathcal{S}}
\newcommand{\HeegaardHandle}{\Hfancy}
\newcommand{\HeegaardSurface}{\Sfancy}
\newcommand{\HeegaardCurve}{\mathbf{c}}
\newcommand{\arrowupgrade}{\twoheadrightarrow}
\newcommand{\norm}[1]{\left\lVert#1\right\rVert}
\newcommand{\vertBand}[2]{B^{v}_{#1, #2}}
\newcommand{\horBand}[2]{B^{h}_{#1, #2}}
\newcommand{\annulus}[1][\epsilon]{A_{#1}}
\newcommand{\Dlinearized}{\mathbf{D}}

\newcommand{\sphere}{\mathbb{S}}
\newcommand{\LagTuple}{\thicc{L}}

\newcommand{\LoopSpace}{\mathcal{L}}
\newcommand{\PathSpace}{\mathcal{P}}

\newcommand{\hypersurface}{\Sigma}
\newcommand{\posRegion}{R^{+}}
\newcommand{\negRegion}{R^{-}}
\newcommand{\posNegRegion}{R^{\pm}}
\newcommand{\Lag}{L}
\newcommand{\posLag}{\Lag^{+}}
\newcommand{\negLag}{\Lag^{-}}
\newcommand{\posNegLag}{\Lag^{\pm}}
\newcommand{\posLeg}{\Leg^{+}}
\newcommand{\negLeg}{\Leg^{-}}
\newcommand{\thiccLeg}{\thicc{\Leg}}
\newcommand{\thiccPosLeg}{\thicc{\Leg}^{+}}
\newcommand{\thiccNegLeg}{\thicc{\Leg}^{-}}
\newcommand{\thiccPosNegLeg}{\thicc{\Leg}^{\pm}}
\newcommand{\posNegLeg}{\Leg^{\pm}}
\newcommand{\divSet}{\Gamma}
\newcommand{\xiDivSet}{\xi_{\divSet}}

%%%%Theorem Definitions%%%%%%%
\newtheorem{thm}{Theorem}[section]
\newtheorem{theorem}[thm]{Theorem}
\newtheorem{claim}[thm]{Claim}
\newtheorem{ex}[thm]{Example}
\newtheorem{assump}[thm]{Assumptions}
\newtheorem*{obs}{Observation}
\newtheorem*{bd}{Boundary Conditions}
\newtheorem{prop}[thm]{Proposition}
\newtheorem{properties}[thm]{Properties}
\newtheorem{param}[thm]{Parameters}
\newtheorem{defn}[thm]{Definition}
\newtheorem{redef}[thm]{Redefinition}
\newtheorem{lemma}[thm]{Lemma}
\newtheorem{cor}[thm]{Corollary}
\newtheorem{conj}[thm]{Conjecture}
\newtheorem{q}[thm]{Question}
\newtheorem{rmk}[thm]{Remark}
\newtheorem{fact}[thm]{Fact}
\newtheorem{facts}[thm]{Facts}
\newtheorem{warn}[thm]{Warning}
\newtheorem{edits}[thm]{Editor notes}
\newtheorem{prob}[thm]{Problem}

%%%% Formatting %%%%%%%%%
%\topmargin0in \textheight8.7in \textwidth6.5in \oddsidemargin0in
%\evensidemargin0in

\title[Weinstein convex hypersurfaces]{A tightness criterion for Weinstein convex hypersurfaces?}
%\author{Russell Avdek}

\begin{document}

\setcounter{tocdepth}{1}
\maketitle
\tableofcontents

\section{What do we want to prove?}

Our goal is to state and prove a tightness criterion for (neighborhoods of) Weinstein convex hypersurfaces \cite{HH:Convex} in analogy with ``a contact manifold is overtwisted iff it is the negative stabilization of some open book''. Such a criterion would be useless to apply against examples, but would be interesting as a theoretical tool and perhaps be useful in developing algebraic invariants which could detect tightness.

Here is our candidate criterion, posed as a definition and question:

\begin{defn}\label{Def:OOB}
We'll say that a bypass as described in \cite[\S 10]{HH:Bypass} is an \emph{obviously overtwisted bypass} (OOB). That is, $\posLeg$ is the unknot, $\posLag$ is the standard Lagrangian disk, and $\posLeg$ is $\theta$-above $\negLeg$.
\end{defn}

\begin{q}\label{Q:OTCriterion}
Is every convex hypersurface $\Sigma$ with an overtwisted $t$-invariant neighborhood the result of an obviously overtwisted bypass attachment to some $\Sigma'$?
\end{q}

To describe convex hypersurfaces we will use Lefschetz fibrations to encode them as \emph{double positive factorizations} (DPFs).\footnote{These are called folded Weinstein Lefschetz fibrations in Breen's most recent article \cite{Breen:Folded}, but these notes are old and I haven't yet had the chance to update notation.} Then bypasses can be described as a sort of combinatorial modification of a DPF. This is not strictly necessary to address a generalized Giroux criterion, but provides a simple means of producing examples in higher dimensions where Legendrian surgeries are difficult to intuit and could give a simpler means of expressing a tightness criterion.

We'll describe what DPFs are, how to express bypasses in the DPF picture, and use them to give explicit examples of obvious overtwisted bypasses (OOBs). We can attach an OOB to the boundary of a Darboux ball, to get an overtwisted $\dim=2n+1$ disk which we call the obviously overtwisted disk (OOD). At the end of the note, we'll give some speculation on how Question \eqref{Q:OTCriterion} might be attacked using OODs. There are some additional notes on how DPFs can be modified by Weinstein Lefschetz fibration moves (as in \cite{BHH:GirouxCorrespondence}), though they are in an even rougher state than the current document.

\section{Double positive factorizations and their geometric realizations}

\subsection{DPFs overview}

Applying \cite{BHH:GirouxCorrespondence, GirouxPardon}, any $\dim = 2n+2$ Weinstein manifold can be described by the ``combinatorial'' data of a Lefschetz fibration, that is, an ordered sequence of (framed) Lagrangian spheres $\LagTuple = (L_{1}, \dots, L_{k})$ in some $\dim=2n$ Weinstein manifold which are the boundaries of Lefschetz thimbles in the total space. Then the boundary is described as an open book with monodromy a product of the Dehn twists along the spheres, $\phi = \tau_{L_{k}}\circ \cdots \circ \tau_{L_{1}}$. An open book with \emph{two} positive Dehn twist factorizations of its monodromy $\phi$ given by some $\LagTuple^{\pm}$ yields two Lefschetz fibrations with their boundaries identified. The total spaces of these Lefschetz fibrations then give the positive and negative regions of a Weinstein convex hypersurface $G(\LagTuple^{\pm})$. 

\begin{defn}
$\LagTuple^{\pm}$ is a \emph{double positive factorization} (DPF) of $\phi$ and the convex hypersurface $G(\LagTuple^{\pm})$ is its \emph{geometric realization}. Write $W$ for the smooth $2n+2$ manifold underlying $G(\LagTuple^{\pm})$.
\end{defn}

By \cite{BHH:GirouxCorrespondence} all Weinstein convex hypersurfaces arise in this way and we know that they are all related by ``WLF moves''. This is a ``Giroux correspondence'' for $\dim > 2$ Weinstein convex hypersurfaces.

\begin{comment}
\begin{ex}
Suppose that $\mu$ and $\lambda$ are meridian and longitude circles on a punctured torus $T \setminus \{ pt \}$. Then $\LagTuple^{+} = (\mu, \lambda, \mu)$ and $\LagTuple^{-} = (\lambda, \mu, \lambda)$ give a DPF, determining a $\dim=4$ convex hypersurface. Indeed $\tau_{\mu}\tau_{\lambda}\tau_{\mu} = \tau_{\lambda}\tau_{\mu}\tau_{\lambda}$ is the classical braid relation. This can be generalized to higher dimensions by looking at matching paths in $A_{3}$ Milnor fibers.
\end{ex}
\end{comment}

\subsection{Some details}

Let $\thicc{\Sigma} = (\Sigma, \beta, f)$ be a Weinstein domain. That is,
\be
\item $\Sigma$ is a compact manifold with boundary,
\item $\beta \in \Omega^{1}(\Sigma)$ is such that $d\beta$ is symplectic,
\item the Liouville vector field $X_{\beta}$ defined $d\beta(X_{\beta}, \ast) = \beta$ points transversely out of $\partial \Sigma$, and
\item $X_{\beta}$ is gradient-like for $f$, meaning that there is a $C > 0$ for which $df(X_{\beta}) > C(|df|^{2} + |X_{\beta}|^{2})$ with respect to some metric on $\Sigma$.
\ee
Consequently $\Sigma$ has even dimension $2n$, and we assume throughout that $n \geq 1$.

A \emph{$\disk$-framed Lagrangian sphere} is a Lagrangian sphere $L \subset \Sigma$ with an identification $L = \sphere^{n} = \partial \disk^{n+1}$ which we call a $\disk$-framing. Two $\disk$ framings are equivalent if they differ by a diffeomorphism of $\sphere^{n}$ which extends to a diffeomorphism of $\disk^{n+1}$. In particular a $\disk$-framing encodes an orientation of $L$ once we have fixed an orientation of $\disk^{n+1}$. Note that an orientation is equivalent to a homotopy class of $\disk$-framing for $n < 6$. To simplify notation, $\disk$-framed Lagrangian spheres will simply denoted $L$ with a $\disk$-framing implicitly specified. We assume that $\disk$-framed Lagrangian spheres are \emph{Legendrian realizable}, meaning that $L$ does not bound a codimension $0$ submanifold of $\Sigma$. This is only a non-trivial constraint when $n > 1$.

Write $\Symp_{c}(\thicc{\Sigma})$ for the space of compactly supported symplectomorphisms of $(\Sigma, d\beta)$. For $\phi, \psi \in \Symp_{c}(\thicc{\Sigma})$, we write $\phi \sim_{0} \psi$ if they live in the same connected component. Clearly $\sim_{0}$ defines an equivalence relation and we write $[\phi]$ for the equivalence class of $\phi$. Note that $\phi \in \Symp_{c}(\thicc{\Sigma})$ applied to a $\disk$-framed Lagrangian is also a $\disk$-framed Lagrangian. The $\beta$ and $f$ parts of a Weinstein structure are transformed by pullback $\phi^{\ast}$ so that compactly supported symplectomorphisms take Weinstein domains to Weinstein domains.

A \emph{positive factorization of $\phi \in \Symp_{c}(\Sigma)$} is an ordered collection of $\disk$-framed Lagrangian spheres $\LagTuple = (L_{1}, \dots, L_{k})$ in $\Sigma$ such that
\begin{equation*}
\tau_{\LagTuple} = \tau_{L_{k}} \circ \cdots \circ \tau_{L_{1}} \sim_{0} \phi
\end{equation*}
where $\tau_{L}$ is a positive symplectic Dehn twist about a Lagrangian sphere $L$. Positive Dehn twists are only defined up to $\sim_{0}$. We abbreviation PF for positive factorization. An \emph{isotopy} of a PF is a modification of one of the $L_{i} \in \LagTuple$ by a symplectic isotopy. Isotopies leave $[\tau_{\LagTuple}]$ unaffected. Observe that $\emptyset$ is a positive factorization of $\Id_{\Sigma}$.

\subsection{Positive factorizations}

A positive factorization $\LagTuple$ of a $\phi \in \Symp_{c}(\Sigma)$ determines a Lefschetz fibration $\pi: W \rightarrow \disk$ whose non-singular fiber is $\thicc{\Sigma} = \pi^{-1}(1)$, with $\# \LagTuple$ singular fibers. Here $\disk$ is the unit disk in $\C$. At the $k$th single fiber $\pi^{-1}(k\epsilon i)$ (with $\epsilon > 0$ small), we have a Lefschetz thimble whose projection to $\disk$ is the straight arc from $k\epsilon i$ to $1$ and whose boundary is $L_{k}$. We say that $(W, \pi)$ is the \emph{geometric realization} of $\LagTuple$. According to \cite{BHH:GirouxCorrespondence, GirouxPardon} every Weinstein manifold is the geometric realization of some positive factorization.

The total space of $W$ then has a Weinstein structure which depends only on the symplectic isotopy class of $\LagTuple$ and the constant $\epsilon$. Then $\partial W$ is naturally a contact manifold and is supported by an open book decomposition with page $(\Sigma, \beta)$ and monodromy $\tau_{\LagTuple}$.

\subsection{Double positive factorizations}

A \emph{double positive factorization} (DPF) is a triple $(\thicc{\Sigma}, \LagTuple^{-}, \LagTuple^{+})$ consisting of a Weinstein manifold $\thicc{\Sigma}$ and a pair ordered tuples of $\disk$-framed Lagrangian spheres $\LagTuple^{\pm} = (L^{\pm}_{i})_{i=1}^{k^{\pm}}$ for which
\begin{equation*}
\tau_{\LagTuple^{-}} \sim_{0} \tau_{\LagTuple^{+}}.
\end{equation*}
There are many famous examples of DPFs: Consider the lantern or chain relations for $\Sigma$ an oriented surface with boundary.

The \emph{geometric realization} of a DPF is a convex hypersurface $W$ of dimension $\dim \Sigma + 2$ whose positive region $W^{+}$ is the geometric realization of $\LagTuple^{+}$ and whose negative region $W^{-}$ is $\LagTuple^{-}$. The boundaries of the $W^{\pm}$ are identified using the open book decompositions with page $\Sigma$ and monodromy $\phi = \tau_{\LagTuple^{\pm}}$.

\begin{thm}
Every convex hypersurface of dimension $\geq 4$ whose positive and negative regions are Weinstein is the geometric realization of some DPF.
\end{thm}

\begin{proof}[Sketch of the proof]
See \cite{Breen:Folded}. Suppose we're given a Weinstein convex hypersurface. Apply \cite{BHH:GirouxCorrespondence, GirouxPardon} to describe the positive and negative regions as geometric realizations of positive factorizations. This gives two open book decompositions of the dividing set $\Gamma$. Applying sufficiently many positive stabilizations to the two open books, they will eventually become isotopic \cite{BHH:GirouxCorrespondence}. The positive stabilization of the open book can be realized as stabilizations of the Lefschetz fibrations on the positive and negative regions, leaving the homotopy classes of their Weinstein structures unmodified. Once the open books are isotopic, we have realized the convex hypersurface as a DPF.
\end{proof}

\section{Bypasses}

\begin{defn}
Bypass attachment data consists of a tuple $(\posNegLeg, \posNegLag)$ for which $\posNegLeg$ is a pair of Legendrian spheres in $\divSet$ intersecting in a single point and $\posNegLag$ are Lagrangian slice disks in the $\posNegRegion$ for the $\posNegLeg$.
\end{defn}

If we attach a bypass to $[-\epsilon, 0] \times \hypersurface$ at $\{ 0 \} \times \hypersurface$ using the data of $(\posNegLeg, \posNegLag)$ then we get a contact structure on $[-\epsilon, 1] \times \hypersurface$ with $\hypersurface_{1} = \{1\} \times \hypersurface$ described as in \cite[Theorem 5.1.3]{HH:Bypass}:
\be
\item $\posNegRegion_{1}$ is obtained by attaching a Weinstein handle to $\negLeg \uplus \posLeg$ and cutting out standard neighborhoods of the $F^{\mp \epsilon}\posNegLag$.
\item $\divSet_{1}$ is obtained by performing a contact $+1$ surgery on $F^{\mp\epsilon}\posNegLeg$ and a contact $-1$ surgery along $(\negLeg \uplus \posLeg)$.
\item The boundaries of the $\posNegRegion_{1}$ are identified by the contactomorphism $\partial \posRegion_{1} \rightarrow \partial \negRegion_{1}$ obtained by sliding $F^{-\epsilon}\posLeg$ up over $(\negLeg \uplus \posLeg)$ to $F^{\epsilon}\negLeg$.
\ee

Briefly, let's describe what is $(\negLeg \uplus \posLeg)$.

\subsection{Legendrian sum}

Given a pair $\posNegLeg$ of Legendrians in a contact manifold $\divSet$ intersecting $\xi$-transversely at a single point, we define the Legendrian sum using the Lagrangian projection formulation from \cite[\S 4.1.2]{HH:Bypass}. Let $W$ be a plumbing of two cotangent bundles of spheres. We can see $W$ sitting inside of $\Mxi$ as a Liouville hypersurface, with the $\posNegLeg$ being zero sections. So when $\dim \posNegLeg = 1$, $W$ is a genus one surface with one boundary component, and the $\posNegLeg$ are circles meeting transversely at one point. We'll later generalize this to higher dimensions. The hypersurface has a neighborhood of the form $[0, 3\epsilon]_{t} \times W$ along which $\alpha = dt + \beta$ for a Liouville form on $W$.

 If the $\posNegLeg$ sit on $\{ \epsilon \} \times W$, then we'll get $\negLeg \uplus \posLeg$ to be the Legendrian lift of the Lagrangian sphere
\begin{equation*}
 \tau_{\posLeg} \negLeg = \tau_{\negLeg}^{-1} \posLeg \subset W
\end{equation*}
where $\tau$ is a Dehn twist. Then
\begin{equation*}
F^{\pm\epsilon}(\negLeg \uplus \posLeg) := \Flow_{\partial_{t}}^{\pm \epsilon}(\negLeg \uplus \posLeg) \subset ([0, \epsilon] \cup [2\epsilon, 3\epsilon]) \times W.
\end{equation*}
After a slight perturbation of the open book the $F^{\pm \epsilon}(\negLeg \uplus \posLeg)$ can be assumed to sit on pages as well.

When $\posLeg$ is a standard unknot, we say that $\posLeg$ is \emph{$\theta$-above} $\negLeg$ if it has a $\theta$ disk which is disjoint from $F^{-\epsilon}\negLeg$ for $\epsilon$ small. See \cite{HH:Bypass} for a definition of $\theta$ disk.

\subsection{Bypasses for DPFs}

\begin{lemma}\label{Lemma:OBBypass}
Suppose that the convex hypersurface $\Sigma = \Sigma_{0}$ is described by a DPF $\thiccPosNegLeg$ with the $\posNegLeg_{1}$ intersecting transversely at a single point in $W$. Then the $\posNegLeg_{1}$ together with their thimbles $\posNegLag_{1} \subset \posNegRegion$ give bypass attachment data. The result of the bypass, $\Sigma_{1}$, can be described by a DPF $\thiccPosNegLeg_{b}$ given by
\begin{equation*}
\thiccPosLeg_{b} = (\tau_{\posLeg_{1}}^{2}\negLeg_{1}, \posLeg_{2}, \dots, \posLeg_{k^{+}}), \quad \thiccNegLeg_{b} = (\tau_{\posLeg_{1}}\negLeg_{1}, \negLeg_{2}, \dots, \negLeg_{k^{-}})
\end{equation*}
\end{lemma}

Let's first verify that the monodromies of the $\thiccPosNegLeg_{b}$ agree as a sanity check:
\begin{equation*}
\begin{aligned}
\tau_{\thiccPosLeg_{b}} &= \tau_{\posLeg_{k^{+}}} \circ \cdots \circ \tau_{\posLeg_{2}} \circ \tau_{\tau_{\posLeg_{1}}^{2}\negLeg_{1}} = \tau_{\posLeg_{k}} \circ \cdots \circ \tau_{\posLeg_{2}} \circ \tau_{\posLeg_{1}} \circ \tau_{\tau_{\posLeg_{1}}\negLeg_{1}} \circ \tau_{\posLeg_{1}}^{-1} \\
&=  \tau_{\thiccPosLeg} \circ \tau_{\tau_{\posLeg_{1}}\negLeg_{1}} \circ \tau_{\posLeg_{1}}^{-1} = \tau_{\thiccNegLeg} \circ \tau_{\tau_{\posLeg_{1}}\negLeg_{1}} \circ \tau_{\posLeg_{1}}^{-1}  = \tau_{\thiccNegLeg} \circ \tau_{\tau_{\negLeg_{1}}^{-1}\posLeg_{1}} \circ \tau_{\posLeg_{1}}^{-1}\\
&= \tau_{\thiccNegLeg} \circ \tau_{\negLeg_{1}}^{-1}\circ \tau_{\posLeg_{1}}\circ \tau_{\negLeg_{1}} \circ \tau_{\posLeg_{1}}^{-1} = \tau_{\negLeg_{k^{-}}}\circ \cdots \circ \tau_{\negLeg_{2}} \circ \tau_{\negLeg_{1}} \circ \tau_{\negLeg_{1}}^{-1}\circ \tau_{\posLeg_{1}}\circ \tau_{\negLeg_{1}} \circ \tau_{\posLeg_{1}}^{-1}\\
&= \tau_{\negLeg_{k^{-}}}\circ \cdots \circ \tau_{\negLeg_{2}} \circ \tau_{\posLeg_{1}}\circ \tau_{\negLeg_{1}} \circ \tau_{\posLeg_{1}}^{-1} = \tau_{\negLeg_{k^{-}}}\circ \cdots \circ \tau_{\negLeg_{2}} \circ \tau_{\tau_{\posLeg_{1}}\negLeg_{1}} = \tau_{\thiccNegLeg_{b}}.
\end{aligned}
\end{equation*}


\begin{figure}[h]
\hspace{-10mm}
\begin{overpic}[scale=.5]{ob_bypass.eps}
\put(100, 0){$\{\epsilon\} \times W$}
\put(100, 12){$\{2\epsilon\} \times W$}
\put(100, 24){$\{3\epsilon\} \times W$}
\put(45, -5){$\partial \posRegion_{1}$}
\put(85, -5){$\partial \negRegion_{1}$}
\put(14, 21){$\negLeg_{1}$}
\put(-1, 15){$\posLeg_{1}$}
\put(32, 0){$(+1)$}
\put(32, 12){$(-1)$}
\put(32, 24){$(-1)$}
\put(70, 0){$(-1)$}
\put(70, 12){$(+1)$}
\put(70, 24){$(-1)$}
\end{overpic}
\vspace{5mm}
\caption{Depiction of bypass surgeries in an open book. Each ``sheet'' is a neighborhood of the point $\posLeg_{1} \cap \negLeg_{1}$ in $W$.}
\label{Fig:OBBypass}
\end{figure}

To prove the lemma we apply the definition of the bypass. Suppose that the $\posNegLeg_{1}$ lie on $\{ 3\epsilon \} \times W$. The cocores of the surgery handles can then be pushed down to $\{2 \epsilon\} \times W$ by the negative Reeb flow as shown in the left-hand side of Figure \ref{Fig:OBBypass} so that they share a single transverse intersection.

Working out $\thiccNegLeg_{b}$ is very easy. According to the definition of the bypass, we must perform a contact $+1$ surgery on a copy of $\negLeg_{1}$ placed slightly above the new surgery locus $\negLeg_{1} \uplus \posLeg_{1} = \tau_{\posLeg_{1}}\negLeg_{1}$ in the boundary of $\partial \negRegion_{1}$. See the right-hand side of Figure \ref{Fig:OBBypass}. The $\pm 1$ surgeries along the $\negLeg_{1}$ at heights $2\epsilon$ and $3\epsilon$ cancel, so that we are only left with the new $-1$ surgery locus, obtaining $\negLeg_{b}$ as claimed.

To describe the boundary of the new positive region $\posRegion_{1}$ we perform a contact $-1$ surgery along $\tau_{\posLeg_{1}}\negLeg_{1}$ (at height $2\epsilon$) along with a contact $+1$ surgery slightly below it, say at $\{\epsilon\} \times W$. See the center row of Figure \ref{Fig:OBBypass}. We want to make the $\pm 1$ surgeries on $\posLeg_{1}$ cancel. To that end, we handle-slide $\negLeg_{1} \uplus \posLeg_{1} = \tau_{\posLeg_{1}}\negLeg_{1}$ (currently at height $2\epsilon$) down through the surgery locus at $\{\epsilon\} \times  \posLeg_{1}$. The before and after pictures are as shown in Figure \ref{Fig:OBBypassSlide}. Because we are flowing down through a $+1$ surgery handle, the effect is a positive Dehn twist: We obtain $\tau_{\posLeg_{1}}^{2}\negLeg_{1}$ on the bottom (say, at height $\epsilon$) and canceling $\pm1$ surgeries (say at heights $2\epsilon$, $3\epsilon$). Deleting the canceling $\pm 1$ surgeries, we are left with $\thiccPosLeg_{b}$ as claimed.

\begin{figure}[h]
\begin{overpic}[scale=.5]{ob_bypass_slide.eps}
\put(-10, 0){$(+1)$}
\put(-10, 19){$(-1)$}
\put(-10, 38){$(-1)$}
\put(52, 0){$(-1)$}
\put(52, 19){$(+1)$}
\put(52, 38){$(-1)$}
\end{overpic}
\vspace{5mm}
\caption{Handle-sliding the new $-1$ surgery locus on $\partial \posRegion_{1}$ down through the new $+1$ surgery locus. The before picture is on the left, and in the after picture on the right $\tau_{\posLeg_{1}}^{2}\negLeg$ sits on the bottom sheet $\{ \epsilon \} \times W$.}
\label{Fig:OBBypassSlide}
\end{figure}

\section{Obviously overtwisted objects}\label{Sec:OOExample}

The definition of the obviously overtwisted bypass (OOB) ensures that the convex hypersurface $\Sigma_{1}$ resulting from the bypass will have an overtwisted neighborhood. It is a priori stronger than the \emph{overtwisted bypass} of \cite{HH:Bypass}. We'll describe an OOB attachment on $\sphere^{2n}$ the boundary of a contact $\dim=2n+1$ Darboux disk, whence $R^{\pm} = \disk^{2n}$. When this OOB is attached to the boundary of the Darboux ball, we get an obviously overtwisted disk (OOD).

\begin{figure}[h]
\begin{overpic}[scale=.25]{front_surface_matching_path.eps}
\end{overpic}
\caption{In each subfigure we see $x$ in blue and $y$ in red. On the left they are boundaries of Lefschetz thimbles in $\Sthree$. In the middle left, a Gompf diagram \cite{Gompf:Handlebodies}, they are Legendrians in the front projection with the skeleton of $W$ in black as in \cite{Avdek:ContactSurgery}. Here $x$ and $y$ are shifted off of the $1$-skeleton (thin black graph) so that $x$ lives on $\{-\epsilon\} \times W$ and $y$ lives on $\{ \epsilon \} \times W$ and the dashed vertical lines represent $1$-handles. In the center-right the Legendrians give a meridian and longitude on a punctured torus. We identify the sides of the square with each other and the top with the bottom, taking the puncture to be the corner. On the right are matching paths for $x,y$ in the $\pi_{W}$ projection.}
\label{Fig:FrontSurfaceMatchingPath}
\end{figure}

\subsection{The OOD}\label{Sec:OOD}

For the Lefschetz fibrations on $\disk^{2n}$ (on which we take standard complex coordinates $z_{k}$) we can use the function $\pi = \delta + z_{1}^{3} + \sum_{2}^{n} z_{k}^{2}$ with $\delta \in \C \setminus 0$ small. Then a regular fiber $W = W^{2n-2}$ will be a plumbing of two cotangent bundles of spheres $x, y$ and the monodromy is $\tau_{y}\tau_{x}$. So we can say that 
\begin{equation*}
\thiccPosNegLeg = (x, y)
\end{equation*}
for both $\pm$. Both of the $x$ and $y$ are unknots in $\disk^{2n}$. According to our convention for ordering Dehn twists, $y \subset W \times \{ \epsilon\}$ while $x \subset W \times \{ -\epsilon\}$.

When $n=2$ the Legendrians can be seen as a meridian and a longitude on a once-punctured torus. To get the monodromy in this case, we can apply \cite[Theorem 4.8]{Avdek:ContactSurgery} and to the left-hand side of Figure \ref{Fig:FrontSurfaceMatchingPath}.

In general, a linear function (eg. $\pi_{W} = z_{1}|_{W}$) gives a Lefschetz fibration on $W$ with three singular fibers, which after an isotopy we can take to be $1, 0, -1 \in \C$ and with non-singular fiber a $\disk^{\ast}\sphere^{n-2}$. Then we can see $x$ as the matching path sphere for the arc $[0, 1] \subset \R \subset \C$ and $y$ as the matching path sphere for the arc $[-1, 0]$. See the right-hand side of Figure \ref{Fig:FrontSurfaceMatchingPath}. In the $n=2$ case, we see that $x$ is $\theta$-above $y$ and we'll assume that this is the case as well in higher dimensions (although I'd have to think about how to prove it).

\begin{figure}[h]
\begin{overpic}[scale=.25]{front_surface_matching_path_mod_2.eps}
\end{overpic}
\caption{The Legendrians $(y, \tau_{y}x)$ shown in green and purple, respectively.}
\label{Fig:FrontSurfaceMatchingPathMod2}
\end{figure}

We can handle-slide $x$ up through $y$ to obtain another positive factorization for $\disk^{2n}$ given by $(y, \tau_{y}x)$ as seen in Figure \ref{Fig:FrontSurfaceMatchingPathMod2}. So
\begin{equation}\label{Eq:BypassSetup}
\thiccPosLeg = (x, y), \quad \thiccNegLeg = (y, \tau_{y}x)
\end{equation}
is a double positive factorization for the standard $\sphere^{2n}$ for which the first Legendrians $\posLeg = x$ and $\negLeg = y$ give OOB data, since $x$ is $\theta$-above $y$.

\begin{defn}\label{Def:OOD}
The \emph{obviously overtwisted disk} (OOD) is the disk with convex boundary $\Sigma_{ot}$ obtained by attaching an obviously overtwisted bypass to the boundary of a standard Darboux disk.
\end{defn}

\subsection{Result of the OOB attachment}\label{Sec:OOBAttchment}

After the bypass is performed, we apply Lemma \ref{Lemma:OBBypass} to see
\begin{equation*}
\thiccPosLeg_{b} = (\tau_{x}^{2}y, y), \quad \thiccNegLeg_{b} = (\tau_{x}y, \tau_{y}x)
\end{equation*}
giving a DPF on the boundary of the OOD.

\begin{figure}[h]
\begin{overpic}[scale=.2]{front_surface_matching_path_post_bypass.eps}
\end{overpic}
\caption{Lefschetz fibrations for the positive region (top) and negative region (bottom) after the obviously overtwisted bypass has been performed.}
\label{Fig:FrontSurfaceMatchingPathPostBypass}
\end{figure}

The bypass data determines a contact structure on $[0, 1] \times \sphere^{2n}$ with $\{0\} \times \sphere^{2n}$ the original hypersurface (given by the boundary of a $\dim=2n+1$ Darboux disk) and
\begin{equation*}
\Sigma_{ot} = \{1\} \times \sphere^{2n}
\end{equation*}
the result of the bypass. The positive and negative regions for $\Sigma_{ot}$ are denoted $R^{\pm}_{ot}$.

A Lefschetz fibrations for $R^{+}_{ot}$ appears in the top row of Figure \ref{Fig:FrontSurfaceMatchingPathPostBypass} with $\tau_{x}^{2}y$ in blue and $y$ in red. On the bottom row of the figure we have a Lefschetz fibration for $R^{-}_{ot}$ with $\tau_{x}y$ in green and $\tau_{y}x$ in purple.

Knowing that \cite{CM:LegendrianFronts} contained lots of pictures, I took a look and got lucky, finding the Lefschetz fibration described in bottom right of Figure \ref{Fig:FrontSurfaceMatchingPathPostBypass} as \cite[Figure 28]{CM:LegendrianFronts}. The $R^{\pm}_{ot}$ are well-studied in the literature. When $2n=6$ they can be described as affine varieties cut out by a single equation
\begin{equation*}
R^{\pm}_{ot} = \{ x(xy - 1) = z_{1}^{2} + z_{2}^{2} \} \subset \C^{4}_{x, y, z_{1}, z_{2}}
\end{equation*}
(and I believe this could be generalized to other dimensions by adding more $z_{k}$s). The following are known facts for the $\dim=4, 6$ cases: The $R^{\pm}_{ot}$ are not flexible but embed into flexible manifolds of the same dimension. They have symplectic homology $SH = 0$ over $\Z$ but $SH \neq 0$ over a twisted coefficient system \cite{MS:Subflexible}, so their holomorphic curves should be quite complicated.

\begin{comment}
\section{Some speculation}\label{Sec:Speculation}

We'll sketch a strategy for proving that the answer to Question \ref{Q:OTCriterion} is yes. Experts might recognize that this is nonsense or that there is a more elegant strategy for addressing Question \ref{Q:OTCriterion}.

Let's take a fixed $\Sigma$ which has an overtwisted $t$-invariant neighborhood $[0, 1] \times \Sigma$. We'll attempt to prove that $\Sigma$ is the result of an OOB attachment.

The $h$-principle should allow us to find OODs in a $t$-invariant $(0, 1)_{t} \times \Sigma$ since this neighborhood is overtwisted. I don't know how to formalize this. Assuming this is true, there will be OOBs inside $[0 ,1)_{t} \times \Sigma$ which can be attached to $\{ 0 \} \times \Sigma$: Attach one end of a contact $1$-handle to the dividing set of $\{ 0 \} \times \Sigma$ and the other end to the dividing set of the OOD. This is the same as first connecting $\{0\} \times \Sigma$ to the boundary of a Darboux ball -- which gives us back $\Sigma$, unmodified -- and then applying an OOB attachment.

Say we attach such an OOB, $B_{1}$, to $\{ 0 \} \times \Sigma$. Then by \cite[Theorem 1.2.5]{HH:Convex} we can get from $\{ 0 \} \times \Sigma$ to $\{ 1 \} \times \Sigma$ by a sequence of bypass attachments $B_{1}, \cdots, B_{k}$ so that the union of all of the bypass attachment slices will be isotopic to the $t$-invariant $[0, 1]\times \Sigma$.

If we can perform handle-slides and rearrange the orders of attachment so that there are bypasses $B'_{1}, \dots, B'_{k'}$ getting us from $\{ 0 \} \times \Sigma$ back to itself and with $B'_{k'}$ obviously overtwisted, then $\Sigma = \{ 1 \} \times \Sigma$ will be the result of an OOB attachment to $\Sigma'$, which is given by attaching the bypasses $B_{1}', \dots, B_{k'-1}'$, answering Question \ref{Q:OTCriterion} in the affirmative.

The OOB $B_{1}$ is attached along a ball in $\Sigma$, so maybe it's possible that it could be avoided during any handle-slides and changing of the orders of bypasses, guaranteeing $B'_{k'} = B_{1}$. If true, then $\Sigma$ would be determined by the model OOB attachment described in \S \ref{Sec:OOExample}. Perhaps this is overly optimistic. Assuming that the strategy of rearranging the orders of bypasses works, I don't know if it would be any easier to prove using DPFs or usual contact handle attachments.
\end{comment}

\begin{thebibliography}{}
\begin{comment}

\bibitem[Abb11]{Abbas:JBook}
C. Abbas, \textit{Holomorphic open book decompositions}, Duke Math. J., Vol. 158, p.29-82, 2011.

\bibitem[AbbCH05]{ACH:PlanarWeinstein}
C. Abbas, K. Cieliebak, H. Hofer, \textit{The Weinstein conjecture for planar contact structures in dimension three}, Comment. Math. Helv. 80, p.771–793, 2005.

\bibitem[AS06]{AS:Cotangent}
A. Abbondandolo and M. Schwarz, \textit{On the Floer homology of cotangent bundles}, Comm.
Pure Appl. Math., 59, p.254–316, 2006.

\bibitem[Abo15]{Abouzaid:Viterbo}
M. Abouzaid, \textit{Symplectic cohomology and Viterbo’s theorem}, In ``Free loop
spaces in geometry and topology'', volume 24 of IRMA Lect. Math. Theor. Phys., Eur. Math. Soc., Z\"{u}rich, p.271–485, 2015

\bibitem[Al79]{Ahlfors}
L. Ahlfors, \textit{Complex Analysis: An Introduction to the Theory of Analytic Functions of One Complex Variable}, New York: McGraw-Hill, 1979.


\bibitem[AS08]{AS:Pencil}
D. Auroux and I. Smith, \textit{Lefschetz pencils, branched covers and symplectic invariants}, in \textit{Symplectic 4-manifolds and algebraic surfaces}, Lecture Notes in Math vol. 1938, Springer, Berlin, p.1–53, 2008.
\end{comment}

\bibitem[Av11]{Avdek:ContactSurgery}
R. Avdek, \textit{Contact surgery and supporting open books}, Algebr. Geom. Topol. Volume 13, p.1613-1660, 2013.

\begin{comment}
\bibitem[Av20]{Avdek:Dynamics}
R. Avdek, \textit{Combinatorial Reeb dynamics on punctured contact $3$-manifolds}, to appear in Geom. Topol., arXiv:2005.11428, 2020.

\bibitem[Av21a]{Avdek:Liouville}
R. Avdek, \textit{Liouville hypersurfaces and connect sum cobordisms}, J. Symplectic Geom., Vol. 19, 2021.

\bibitem[Av21b]{Avdek:LSFT}
R. Avdek, \textit{Simplified SFT moduli spaces for Legendrian links}, arXiv:2104.00505, 2021.

\bibitem[Av22a]{Avdek:PDH}
R. Avdek, \textit{A filtered generalization of the Chekanov-Eliashberg algebra}, arXiv:2205.13031, 2022.

\bibitem[Av22b]{Avdek:Software}
R. Avdek, \textit{Legendrian links}, software available at \url{https://github.com/RAvdek/legendrian_links}, 2022.

\bibitem[BH15]{BH:ContactDefinition}
E. Bao and K. Honda, \textit{Semi-global Kuranishi charts and the definition of contact homology}, arXiv:1512.00580, 2015.

\bibitem[BH18]{BH:Cylindrical}
E. Bao and K. Honda, \textit{Definition of cylindrical contact homology in dimension three}, Journ. of Topology, Vol. 11, No. 4, p.1002-1053, 2018.


\bibitem[BM]{SmoothAtlas}
D. Bar-Natan, S. Morrison, et. al, \textit{The Rolfsen knot table}, \url{http://katlas.org/wiki/The_Rolfsen_Knot_Table}

\bibitem[Bj16]{Bjorklund:LCH}
J. Bj\"{o}rklund, \textit{Legendrian contact homology in the product of a punctured Riemann surface and the real line}, Journal of the London Math. Soc., Vol. 94, no 3, p.970-992, 2016.
\end{comment}

\bibitem[BEM15]{BEM:OT}
M. S. Borman, Y. Eliashberg, and E. Murphy, \textit{Existence and classification of overtwisted contact structures in all dimensions}, Acta Math. 215, p.281–361, 2015.

\begin{comment}
\bibitem[B02]{Bourgeois:Thesis}
F. Bourgeois, \textit{A Morse-Bott approach to contact homology}, PhD thesis, Stanford University, 2002.

\bibitem[B03]{Bourgeois:ContactIntro} 
F. Bourgeois, \textit{Introduction to contact homology}, lecture notes available at \url{https://www.imo.universite-paris-saclay.fr/~bourgeois/papers/Berder.pdf}, 2003.

\bibitem[BC14]{BC:Bilinearized}
F. Bourgeois and B. Chantraine, \textit{Bilinearized Legendrian contact homology and the augmentation category}, J. Symplectic Geom., Volume 12, p.553–583, 2014.

\bibitem[BG19]{BG:Bilinearized}
F. Bourgeois and D. Galant, \textit{Geography of bilinearized Legendrian contact homology}, arXiv:1905.12037, 2019.


\bibitem[BM03]{BM:Orientations}
F. Bourgeois and K. Mohnke, \textit{Coherent orientations in symplectic field theory}, Math. Z. 248, p.123-146, 2003.

\bibitem[BEE11]{BEE:Product}
F. Bourgeois, T. Ekholm, and Y. Eliashberg, \textit{Symplectic homology product via Legendrian surgery}, PNAS Volume 108, Number 20, p.8114-8121, 2011.

\bibitem[BEE12]{BEE:LegendrianSurgery}
F. Bourgeois, T. Ekholm, and Y. Eliashberg, \textit{Effect of Legendrain Surgery}, Geom. Topol., Volume 16, Number 1, p.301-389, 2012.

\bibitem[BEHW03]{SFTCompactness}
F. Bourgeois, Y. Eliashberg, H. Hofer, and K. Wysocki, \textit{Compactness results in symplectic field theory}, Geom. Topol., Volume 7, Number 2, p.799-888, 2003.

\bibitem[BN10]{AlgebraicallyOvertwisted}
F. Bourgeois and K. Niederkr{\"u}ger, Towards a good definition of algebraically overtwisted,
Expo. Math. 28, no. 1, 85–100, 2010.

\bibitem[BO09a]{BO:MorseBott}
F. Bourgeois and A. Oancea, \textit{Symplectic homology, autonomous Hamiltonians, and Morse–Bott moduli spaces}, Duke Math. J. 146, p.71–174, 2009.

\bibitem[BO09b]{BO:ExactSequence}
F. Bourgeois and A. Oancea, \textit{An exact sequence for contact- and symplectic homology},
Invent. math., Volume 175, p. 611–680, 2009.

\bibitem[BT82]{BottTu}
R. Bott and L. R. Tu, \textit{Differential forms in algebraic topology}, Graduate Texts in Mathematics 82, Springer-Verlag, 1982.
\end{comment}

\bibitem[Br23]{Breen:Folded}
J. Breen, \textit{Folded symplectic forms in contact topology}, arXiv:2311.16058, 2023.

\bibitem[BHH]{BHH:GirouxCorrespondence}
J. Breen, K. Honda, and Y. Huang, \textit{The Giroux correspondence in arbitrary dimensions}, arXiv:2307.02317, 2023.

\begin{comment}
\bibitem[CGKS14]{CGKS:Polyfilling}
C. Cao, N. Gallup, K. Hayden, J. M. Sabloff, \textit{Topologically Distinct Lagrangian and Symplectic Fillings}, Math. Research Letters, Vol. 21, p.85-99, 2014.

\bibitem[CMP19]{CMP:OT}
R. Casals, E. Murphy, and F. Presas, \textit{Geometric criteria for overtwistedness}, J. Amer. Math. Soc. 32, p.563–604, 2019.
\end{comment}

\bibitem[CM19]{CM:LegendrianFronts}
R. Casals and E. Murphy, \textit{Legendrian fronts for affine varieties}, Duke Math. J. 168(2): p.225-323, 2019.


\begin{comment}
\bibitem[CG86]{CG:Cobordisms}
A. Casson, C. M. Gordon, \textit{Cobordism of classical knots}, from: "{\'A} la recherche de la topologie perdue" (editors L Guillou, A Marin), Progr. Math. 62, Birkh{\"a}user 181, 1986.

\bibitem[CDGG20]{Cthulu}
B. Chantraine, G. Dimitroglou Rizell, P. Ghiggini, and R. Golovko, \textit{Floer theory for Lagrangian cobordisms}, J. Differential Geom. 114 no. 3, p.393–465, 2020.

\bibitem[Ch10]{Chantraine:Concordance}
B. Chantraine, \textit{Lagrangian concordance of Legendrian knots}, Algebr. Geom. Topol. 10, p.63-85, 2010.

\bibitem[Ch10]{Chantraine:Symmetric}
B. Chantraine, \textit{Lagrangian concordance is not a symetric relation}, Quantum Topol. Vol. 6,, p.451-474, 2015.

\bibitem[CS99]{StringTopology}
M. Chas and D. Sullivan \textit{String topology}, preprint (1999), math.GT/9911159

\bibitem[CS04]{CS:StringDiagrams}
M. Chas, and D. Sullivan, \textit{Closed string operators in topology leading to Lie bialgebras and higher string algebra}, The legacy of Niels Henrik Abel, p.771–784, Springer, Berlin, 2004.

\bibitem[C02]{Chekanov:LCH}
Y. Chekanov, \textit{Differential algebra of Legendrian links}, Invent. Math., Volume 150, Number 3, p.441–483, 2002.

\bibitem[CN15]{LegKnotAtlas}
W. Chongchitmate and L. Ng, \emph{The Legendrian knot atlas},
\url{http://www.math.duke.edu/~ng/atlas/}, 2015.

\bibitem[Ci02a]{Cieliebak:Subcritical}
K. Cieliebak, \textit{Subcritical Stein manifolds are split}, arXiv:math/0204351v1, 2002.

\bibitem[Ci02b]{Cieliebak:SubcriticalSH}
K. Cieliebak, \textit{Handle attaching in symplectic homology and the Chord Conjecture}, J.
Eur. Math. Soc., p.115–142, 2002.



\bibitem[CE12]{SteinToWeinstein} 
K. Cieliebak and Y. Eliashberg, \textit{From Stein to Weinstein and back}, volume 59 of American
Mathematical Society Colloquium Publications. American Mathematical Society, Providence, RI,
2012.

\bibitem[CFL20]{CFL:Linfty}
K. Cieliebak, K. Fukaya, and J. Latschev, \textit{Homological algebra related to surfaces with boundary}, Quantum Topol., Vol. 11, p.691–837, 2020.

\bibitem[CL07]{CL:SFTStringTop}
K. Cieliebak and J. Latschev, \textit{The role of string topology in symplectic field theory},  New perspectives and challenges in symplectic field theory, CRM Proc. Lecture Notes, vol. 49, Amer. Math. Soc., p. 113–146, 2009.

\bibitem[CM07]{CM:GW}
K. Cieliebak and K. Mohnke, \textit{Symplectic hypersurfaces and transversality in Gromov-Witten theory}, J. Symplectic Geom. 5, p.281-356, 2007.

\bibitem[CEKSW11]{CEKSW:AugCat}
G. Civan, J. Etnyre, P. Koprowski, J. Sabloff, and A. Walker, \textit{Product structures for Legendrian contact homology}, Math. Proc. Cambridge Philos. Soc., Vol. 150, p.291–311, 2011.

\bibitem[CGH11]{CGH:HFequalsECH}
V. Colin, P. Ghiggini and K. Honda, \textit{Equivalence of Heegaard Floer homology and embedded contact homology via open book decompositions}, Proceedings of the National Academy of Sciences of the United States of America,
Vol. 108, No. 20, p.8100-8105, 2011

\bibitem[CGHH10]{CGHH:Sutures}
V. Colin, P. Ghiggini, K. Honda, and M. Hutchings, \textit{Sutures and contact homology I}, Geom. Topol., Vol. 15, p.1749-1842, 2011.


\bibitem[CHT]{CHT:HF}
V. Colin, K. Honda, Y. Tian, \textit{Applications of higher-dimensional Heegaard Floer homology to contact topology}, https://arxiv.org/abs/2006.05701, 2020.

\bibitem[CET19]{CET:PlusOneFilling}
J. Conway, J. Etnyre, and B. Tosun, \textit{Symplectic fillings, contact surgeries, and Lagrangian disks}, IMRN, 2019.


\bibitem[CKK14]{CKK:TightPlanar}
J. Conway, A. Kaloti, D. Kulkarni, \textit{Tight Planar Contact Manifolds with Vanishing Heegaard Floer Contact Invariants}, arXiv:1409.4055, 2014.

\bibitem[CFC20]{CFC:RelativeContactHomology}
L. C\^{o}t\'{e}, F.S. Fauteux-Chapleau, \textit{Homological invariants of codimension 2 contact submanifolds}, 	arXiv:2009.06738, 2020.


\bibitem[CNS16]{CNS:Concordance}
C. Cornwell, L. Ng, and S. Sivek, \textit{Obstructions to Lagrangian concordance}, Algebr. Geom. Topol., Vol. 16, p.797–824, 2016.

\bibitem[DL19]{DiogoLisi}
L. Diogo and S. Lisi, \textit{Symplectic homology of complements of smooth divisors}, J. Topology 12, p. 967-1030, 2019. 

\bibitem[DE20]{DE:Alexander}
L. Diogo and T. Ekholm, \textit{Augmentations, annuli, and Alexander polynomials}, 	arXiv:2005.09733, 2020.

\bibitem[Di92]{Dimca}
A. Dimca, \textit{Singularities and the topology of hypersurfaces}, Springer-Verlag, New York, 1992.

\bibitem[DR16]{DR:Lifting}
G. Dimitroglou Rizell, \textit{Lifting pseudo-holomorphic polygons to the symplectization of $P \times \R$ and applications}, Quantum Topol. 7, p.29–105, 2016.

\bibitem[DRS20]{DRS:Persistence}
G. Dimitroglou Rizell and M. Sullivan, \textit{The persistence of the Chekanov–Eliashberg algebra}, Sel. Math. New Ser., Vol. 26, 2020.

\bibitem[DF18]{DF:HCompactness}
A. Doicu and U. Fuchs, \textit{A Compactness Result for $\mathcal{H}$-holomorphic Curves in Symplectizations}, arXiv:1802.08991, 2018.

\bibitem[DG04]{DG:Surgery}
F. Ding and H. Geiges, \textit{A Legendrian surgery presentation of contact 3-manifolds}, Math. Proc. Cambridge Philos. Soc.,
Vol. 136, p.583-598, 2004.

\bibitem[DG08]{DG:HandleMoves}
F. Ding and H. Geiges, \textit{Handle moves in contact surgery diagrams}, J. Topology, Volume 2, Number 1, p.105-122, 2008.

\bibitem[Dr04]{Dragnev}
D. Dragnev, \textit{Fredholm theory and transversality for noncompact pseudoholomorphic maps in symplectizations}, Comm. Pure Appl. Math. 57, p.726–763, 2004.

\bibitem[EK07]{Ekholm:FlowTrees}
T. Ekholm, \textit{Morse flow trees and Legendrian contact homology in 1-jet spaces}, Geom. Topol., Vol. 11, p.1083–1224, 2007.

\bibitem[Ek08]{Ekholm:Z2RSFT}
T. Ekholm, \textit{Rational symplectic field theory over $\Z_{2}$ for exact Lagrangian cobordisms}, J. Eur. Math. Soc. (JEMS) 10, p.641–704, 2008.

\bibitem[Ek16]{Ekholm:Nonloose}
T. Ekholm, \textit{Non-loose Legendrian spheres with trivial contact homology DGA}, Journal of Topology, Volume 9, p.826–848, 2016.

\bibitem[Ek16]{Ekholm:SurgeryLectures}
T. Ekholm, \textit{Legendrian surgery formulas and applications}, lecture notes from the SFT-VIII conference available at \url{https://www.mathematik.hu-berlin.de/~wendl/SFT8/program.html}, 2016.

\bibitem[Ek19]{Ekholm:SurgeryCurves}
T. Ekholm, \textit{Holomorphic curves for Legendrian surgery}, arXiv:1906.07228, 2019.

\bibitem[EES09]{EES:Duality}
T. Ekholm, J. B. Etnyre, and J. M. Sabloff, \textit{A duality exact sequence for Legendrian contact
homology}, Duke Math. J., 150(1), p.1–75, 2009.

\bibitem[EES05]{EES:Orientations}
T. Ekholm, J. Etnyre, and M. Sullivan, \textit{Orientations in Legendrian Contact Homology and Exact Lagrangian Immersions}, IJM 16, p.453-532, 2005.

\bibitem[EES05]{EES:LegendriansInR2nPlus1}
T. Ekholm, J. Etnyre, and M. Sullivan, \textit{The contact homology of Legendrian submanifolds of $\R^{2n+1}$}, J. Differential Geom. Volume 71, Number 2, p.177-305, 2005.

\bibitem[EHK16]{EHK:LagrangianCobordisms}
T. Ekholm, K. Honda, and T. K\'{a}lm\'{a}n, \textit{Legendrian knots and exact Lagrangian cobordisms}, J. Eur. Math. Soc. (JEMS), 18(11), p.2627–2689, 2016.

\bibitem[EkN15]{EkholmNg:Subcritical}
T. Ekholm and L. Ng, \textit{Legendrian contact homology in the boundary of a subcritical Weinstein 4-manifold}, J. Differential Geom., Volume 101, Number 1, p.67-157, 2015.

\bibitem[EkN18]{EkholmNg:LSFT}
T. Ekholm and L. Ng, \textit{Higher genus knot contact homology and recursion for colored HOMFLY-PT polynomials}, arXiv:1803.04011, 2018.

\bibitem[ENS18]{ENS:CompleteInvariant}
T. Ekholm, L. Ng, and V. Shende, \textit{A complete knot invariant from contact homology}, Invent. Math., Vol. 211, p.1149–1200, 2018.

\bibitem[E98]{Eliashberg:LCH}
Y. Eliashberg, \textit{Invariants in contact topology}, Proceedings of the International Congress of Mathematicians, Vol. II (Berlin), number Extra Vol. II, p.327–338 (electronic), 1998.

\bibitem[EGH00]{EGH:SFTIntro}
Y. Eliashberg, A Givental, and H. Hofer, \textit{Introduction to symplectic field theory}, Geom. Funct. Anal., Special Volume, Part II, p.560-673, 2000.

\bibitem[El89]{Eliash:OTClassification}
Y. Eliashberg, \textit{Classification of overtwisted contact structures on 3-manifolds}, Invent. Math. 98, p.623-637, 1989.

\bibitem[El91]{Eliash:Filling}
Y. Eliashberg, \textit{On symplectic manifolds with some contact properties}, J. Differential Geom. 33, p.233–238, 1991.

\bibitem[EM02]{EM:Hprinciple}
Y. Eliashberg, N. Michachev, {\em Introduction to the h-principle}, Graduate Studies in Mathematics, Volume 48, American Mathematical Society, Providence, RI (2002)

\bibitem[EP96]{EP:UnknotFIlling}
Y. Eliashberg and L. Polterovich, \textit{Local Lagrangian 2-knots are trivial}, Ann. of Math., Vol. 144, p.61–76, 1996.

\bibitem[Et05]{Etnyre:KnotNotes}
J. Etnyre, \textit{Legendrian and Transversal Knots}, Handbook of Knot Theory Elsevier B. V., p.105-185, 2005.

\bibitem[Et08]{Etnyre:ContactSurgery}
J. Etnyre, \textit{On contact surgery}, Proc. Amer. Math. Soc., Volume 136, p.3355-3362, 2008.

\bibitem[EtGh00]{EG:ReebLinks}
J. Etnyre and R. Ghrist, \textit{Contact Topology and Hydrodynamics III: Knotted Orbits}, Transactions of the Amer. Math. Soc., Vol. 352, No. 12, p. 5781-5794, 2000.

\bibitem[EH01]{EtnyreHonda:Knots}
J. Etnyre and K. Honda, \textit{Knots and Contact Geometry I: Torus Knots and the Figure Eight Knot}, J. Symplectic Geom., Volume 1, Number 1, p.63-120, 2001.

\bibitem[EN18]{EtnyreNg:LCHSurvey}
J. Etnyre and L. Ng, \textit{Legendrian contact homology in $\R^{3}$}, arXiv:1811.10966v3, 2018.

\bibitem[ENS02]{ENS:Orientations}
J. Etnyre, L. Ng and J. Sabloff, \textit{Invariants of Legendrian Knots and Coherent Orientations}, J. Symp. Geom., Vol. 1, No. 2, p.321–367, 2002.

\bibitem[ENV13]{ENV:Twists}
J. Etnyre, L. Ng, and V. Vertesi, \textit{Legendrian and transverse twist knots}, J. Eur. Math. Soc. (JEMS), Volume 15, no. 3, p.969–995, 2013.

\bibitem[EO08]{EO:OBInvariants}
J. Etnyre and B. Ozbagci, \textit{Invariants of contact structures from open books}, Trans. Amer. Math. Soc. Volume 360, Number 6, p.3133–3151, 2008.

\bibitem[EV18]{EV:Satellites}
J. Etnyre and V. Vertesi, \textit{Legendrian satellites}, IMRN, Issue 22, p.7241–7304, 2018.

\bibitem[Fa11]{Fabert:Pants}
O. Fabert, \textit{Obstruction bundles over moduli spaces with boundary and the action filtration in symplectic field theory}, Mathematische Zeitschrift, Volume 269, p.325–372, 2011.


\bibitem[Fl88]{Floer:Morse}
A. Floer, \textit{Morse theory for Lagrangian intersections}, J. Diff. Geometry, p.513-547, 1988.


\bibitem[Fu06]{Fukaya:LagrangianSubmanifolds}
K. Fukaya, \textit{Applications of Floer homology of Lagrangian submanifolds to symplectic topology}, Morse Theoretic Methods in Nonlinear Analysis and in Symplectic Topology (P.
Biran, O. Cornea, eds.), Nato Science Series, Volume 217, p.231-276, 2006.

\bibitem[FOOO09]{FOOO}
K. Fukaya, Y. Oh, H. Ohta and K. Ono, \textit{Lagrangian intersection Floer theory: anomaly
and obstruction, Part II}, AMS/IP Studies in Advanced Mathematics, 46.2. American Mathematical
Society, Providence, RI; International Press, Somerville, MA, 2009.

\bibitem[FT17]{FT:Kuranishi}
K. Fukaya and M. Tehrani, \textit{Gromov-Witten theory via Kuranishi structures}, arXiv:1701.07821, 2017.

\bibitem[G94]{Geiges:DisconnectedBoundary}
H. Geiges, \textit{Symplectic manifolds with disconnected boundary of contact type}, Internat. Math. Res. Notices, p.23–30, 1994.


\bibitem[GZ13]{GZ:FourBall}
H. Geiges and K. Zehmisch, \textit{How to recognize a 4-ball when you see one}, M{\"u}nster J. Math., Volume 6, p.525–554, 2013.

\bibitem[G91]{Giroux:Convexity}
E. Giroux, \textit{Convexit\'{e} en topologie de contact}, Comment. Math. Helv. 66, p.637–677, 1991.

\bibitem[G02]{Giroux:ContactOB}
E. Giroux, \textit{G\'{e}om\'{e}trie de contact: de la dimension trois vers les dimensions sup\'{e}rieures}, Proceedings of the International Congress of Mathematicians, Vol. II, Higher Ed. Press, Beijing, p.405-414, 2002.

\bibitem[Girou17]{Giroux:IdealLiouville}
E. Giroux, \textit{Ideal Liouville Domains - a cool gadget}, J. Symplectic Geom., Volume 18, p.769–790, 2017.
\end{comment}

\bibitem[GP17]{GirouxPardon}
E. Giroux and J. Pardon, \textit{Existence of Lefschetz fibrations on Stein and Weinstein domains}, Geom. Topol., 21, no. 2, 963–997, 2017.

\bibitem[Go98]{Gompf:Handlebodies}
R. Gompf, \textit{Handlebody construction of Stein surfaces}, Ann. of Math. 148,
p.619-693, 1998.

\begin{comment}
\bibitem[GS99]{GS:KirbyCalculus}
R. Gopf and A. Stipsicz, \textit{4-manifolds and Kirby Calculus}, Graduate Studies in Mathematics 20, Amer. Math. Society, Providence, RI, 1999.

\bibitem[Gr85]{Gromov:JCurves}
M. Gromov, \textit{Pseudoholomorphic curves in symplectic manifolds}, Invent. Math, Volume 82, p.307–347, 1985.


\bibitem[G14]{Gutt:Normal}
J. Gutt, \textit{Normal Forms for Symplectic Matrices}, Portugalia Mathematicae, vol. 71, p.109-139, 2014.



\bibitem[Ha02]{Hatcher:AlgebraicTopology}
A. Hatcher, \textit{Algebraic topology}, Cambridge University Press, Cambridge, 2002.

\bibitem[HW]{HW:Cop}
N. Higstons, N. Wahl, {\em Products and coproducts in string topology}, arXiv:1709.06839v1, 2017.

\bibitem[Hi03]{Hind:Filling}
R. Hind, \textit{Stein fillings of lens spaces}, Commun. Contemp. Math. 5, no. 6, p.967–982, 2003.

\bibitem[Hof93]{Hofer:OTWeinstein}
H. Hofer, \textit{Pseudoholomorphic curves in symplectizations with applications to the Weinstein conjecture in dimension three}, Inv. Math., Volume 114, p.515-563, 1993.


\bibitem[HWZ96]{HWZ:Asymptotics}
H. Hofer, K. Wysocki and E. Zehnder, \textit{Properties of pseudoholomorphic curves in symplectizations I: Asymptotics}, Ann. Inst. H. Poincar{`e} Anal. Non Lin{`e}aire 13, p.337–37, 1996.

\bibitem[H00]{Honda:Tight1}
K. Honda, \textit{On the classification of tight contact structures I}, Geom. Topol., Volume 4, Number 1, p.309-368, 2000.

\bibitem[Hon00]{Honda:Tight2}
K. Honda, \textit{On the classification of tight contact structures II}, J. Differential Geom., Volume 55, Number 1, p.83-143, 2000.

\bibitem[Hon02]{Honda:OTSurgery}
K. Honda, \textit{Gluing tight contact structures}, Duke Math. J.
Volume 115, Number 3, p.435-478, 2002.

\bibitem[HKM09]{HKM:ContactClass}
K. Honda, W. Kazez, and G. Mati\'{c}, \textit{On the contact class in Heegaard Floer homology}, J. Differential Geom. Volume 83, Number 2, p.289-311, 2009.
\end{comment}

\bibitem[HH18]{HH:Bypass}
K. Honda and H. Huang, \textit{Bypass attachments in higher-dimensional contact topology}, preprint, arXiv:1803.09142, 2018.

\bibitem[HH19]{HH:Convex}
K. Honda and H. Huang, \textit{Convex hypersurface theory in contact topology}, preprint, arXiv:1907.06025, 2019.

\bibitem[HT22]{HT:Category}
Ko Honda and Yin Tian, \textit{Contact categories of disks}, J. Symp. Geom., Vol. 20, 2022.

\begin{comment}
\bibitem[Hua13]{Huang:OTClassification}
Y. Huang, \textit{A proof of the classification theorem of overtwisted contact structures via convex surface theory}, J. Symplectic Geom.
Volume 11, Number 4, p.563-601, 2013.


\bibitem[Hum97]{Hummel}
C. Hummel, \textit{Gromov’s compactness theorem for pseudo-holomorphic curves}, Progress in Mathematics, 151, Birkh\"{a}user Verlag, Basel, 1997.

\bibitem[Hut13]{Hutchings:QOnly}
M. Hutchings, \textit{Rational SFT using only q variables}, blog post available at \url{https://floerhomology.wordpress.com/2013/04/23/rational-sft-using-only-q-variables/}, 2013.

\bibitem[Hut14]{Hutchings:ECHNotes}
M. Hutchings, \textit{Lecture Notes on Embedded Contact Homology}, in Contact and Symplectic Topology, Bolyai Society Mathematical Studies, vol 26, Springer, p.389-484, 2014.

\bibitem[HS06]{HS:T3}
M. Hutchings and M. Sullivan, \textit{Rounding corners of polygons and the embedded contact homology of $T^3$}, Geometry and Topology, Vol. 10, p. 169-266, 2006.


\bibitem[HT07]{HT:GluingI}
M. Hutchings and C. Taubes, \textit{Gluing pseudoholomorphic curves along branched covered cylinders I}, J. Symplectic Geom. 5, p.43–137, 2007.

\bibitem[HT09]{HT:GluingII}
M. Hutchings and C. Taubes, \textit{Gluing pseudoholomorphic curves along branched covered cylinders II}, J. Symplectic Geom. 7, p.29-133, 2009.

\bibitem[Kar20]{Karlsson:Orientations}
C. Karlsson, \textit{A note on coherent orientations for exact Lagrangian cobordisms}, Quantum
Topol., 11(1), p.1–54, 2020.

\bibitem[K87]{Kauffman}
L. H. Kauffman, \textit{State models and the Jones polynomial}, Topology, Volume 26, Number 3, p.395–407, 1987.

\bibitem[Ko00]{Khovanov}
M. Khovanov, \textit{A categorification of the Jones polynomial}, Duke Math. J., Volume 101, Number 3, p.359-426, 2000.

\bibitem[KM93]{KM:MilnorConj}
P. Kronheimer and T. Mrowka. \textit{Gauge theory for embedded surfaces I}. Topology, 32, p.773–826, 1993.

\bibitem[KLT10]{KLT:HFSW}
C. Kutluhan, Y.-J. Lee, and C. Taubes, \textit{HF=HM I: Heegaard Floer homology and Seiberg-Witten Floer homology}, arXiv:1007.1979, 2010.

\bibitem[La93]{Lang142}
S. Lang, \textit{Real and functional analysis}, Graduate texts in mathematics 142, Springer, 1993.

\bibitem[LW11]{LW:Torsion}
J. Latschev and C. Wendl, \textit{Algebraic torsion in contact manifolds (with
an appendix by M. Hutchings)}, Geom. Funct. Anal., Vol. 21, p.1144–1195, 2011.

\bibitem[Li06]{Lipshitz:Cylindrical}
R. Lipshitz. \textit{A cylindrical reformulation of Heegaard Floer homology}, Geometry and Topology, Vol. 10, p.955–1097, 2006.

\bibitem[LOT18]{LOT:Bordered}
R. Lipshitz, P. Ozsv{\'a}th, and D. Thurston, \textit{Bordered Heegaard Floer homology: Invariance and pairing}, Mem. Amer. Math. Soc. 254, no. 1216, 2018.

\bibitem[L98]{Lisca:Nonfillable}
P. Lisca, \textit{Symplectic fillings and positive scalar curvature}, Geom. Topol. 2, p.103-116, 1998.

\bibitem[L08]{Lisca:Filling}
P. Lisca, On symplectic fillings of lens spaces, Trans. Amer. Math. Soc. 360, no. 2, p.765–799, 2008.

\bibitem[LS04]{LS:TightI}
P. Lisca and A. I. Stipsicz, \textit{Ozsváth–Szábo invariants and tight contact three-manifolds I}, Geom. Topol., Volume 8, Number 2, p.925-945, 2004.

\bibitem[LS06]{LS:ContactClassNotes}
P. Lisca and A. I. Stipsicz, \textit{Notes on the Contact Ozsváth-Szabó Invariants}, Pacific J. Math, Volume 228, Number 2, p.277–295, 2006.

\bibitem[L02]{Liu:Moduli}
C.C.M. Liu, \textit{Moduli of J-holomorphic curves with Lagrangian boundary conditions and open Gromov-Witten invariants for an $\Circle$-equivariant pair}, Ph.D. thesis (Harvard University), arXiv:math.SG/0210257, 2002.

\bibitem[Lo02]{Long:Index}
Y. Long, \textit{Index theory for symplectic paths with applications}, volume 207 of
Progress in Mathematics. Birkh\"{a}user Verlag, Basel, 2002.

\bibitem[MNW13]{MNW13}
P. Massot, K. Niederkr\"{u}ger, and C. Wendl, \textit{Weak and strong fillability of higher dimensional contact manifolds}, Invent. math. (2013), 192-287.

\bibitem[M90]{McDuff:RationalRuled}
D. McDuff, \text{The structure of rational and ruled symplectic 4-manifolds}, J. Amer. Math. Soc. 3, p.679–712, 1990.

\bibitem[M91]{McDuff:Filling}
D. McDuff, Symplectic manifolds with contact type boundaries, Invent. Math. 103, p.651–671, 1991.

\bibitem[MS99]{MS:SymplecticIntro}
D. McDuff and D. Salamon, \textit{Introduction to symplectic topology}, Second edition. Oxford Mathematical Monographs. The Clarendon Press, Oxford University Press, New York, 1998.

\bibitem[MS04]{MS:Curves}
D. McDuff and D. Salamon, \textit{J-holomorphic curves and symplectic topology}, American Mathematical Society, Providence, RI, 2004.

\bibitem[MZ20]{MZ:Torsion}
A. Moreno and Z. Zhou, \textit{A landscape of contact manifolds via rational SFT}, 	arXiv:2012.04182, 2020.
\end{comment}

\bibitem[MS18]{MS:Subflexible}
E. Murphy and K. Siegel, \textit{Subflexible symplectic manifolds}, Geom. Topol. 22, p.2367-2401, 2018.

\begin{comment}
\bibitem[N01]{Ng:Satellites}
L. Ng, The Legendrian satellite construction, arXiv:math/0112105, 2001.

\bibitem[N03]{Ng:ComputableInvariants}
L. Ng, \textit{Computable Legendrian invariants}, Topology, Volume 42, p.55–82, 2003.

\bibitem[N10]{Ng:RSFT}
L. Ng, \textit{Rational symplectic field theory for Legendrian knots}, Invent. math., Volume 182, Issue 3, p.451–512, 2010.

\bibitem[N14]{Ng:KCHIntro}
L. Ng, \textit{A topological introduction to knot contact homology}, In Contact and Symplectic Topology, Bolyai Soc. Math. Stud. 26 (Springer, Berlin) 2014.

\bibitem[N]{Ng:Software}
L. Ng et al., \textit{Programs and packages}, \url{https://services.math.duke.edu/~ng/math/programs.html}.

\bibitem[NR13]{NR:Helix}
L. Ng and D. Rutherford, \textit{Satellites of Legendrian knots and representations of the Chekanov–Eliashberg algebra}, Algebr. Geom. Topol., Volume 13, Number 5, p.3047-3097, 2013.

\bibitem[N06]{N:Plastik}
Klaus Niederkr\"{u}ger, \textit{The plastikstufe -- a generalization of the overtwisted disk to higher dimensions}, Algebr. Geom. Topol. 6, p.2473–2508, 2006.


\bibitem[O04]{Oancea:Survey}
A. Oancea, \textit{A survey of Floer homology for manifolds with contact type boundary or symplectic homology}, Ensaios Mat., 7, p.51–91, 2004.

\bibitem[Oh96]{Oh:LagPert}
Y.G. Oh, \textit{Fredholm theory of holomorphic discs under the perturbation of boundary conditions}, Math. Z., Vol. 222, p.505–520, 1996.

\bibitem[OO05]{OO:Filling} 
H. Ohta and K. Ono, \textit{Simple singularities and symplectic fillings}, J. Differential
Geom. 69, no. 1, p.1–42, 2005.

\bibitem[O05]{Ozbagci:Stabilization}
B. Ozbagci, \textit{A note on contact surgery diagrams}, Int. J. Math., Volume 16, Number 1, p.87-99, 2005.

\bibitem[O09]{Ozbagci:CotangentBundle}
B. Ozbagci, \textit{Stein and Weinstein structures on disk cotangent bundles of surfaces}, Archiv der Mathematik, Volume 113, Issue 6, p.661–670, 2019.

\bibitem[OS04]{OS:SurgeryBook}
B. Ozbagci and A. I. Stipsicz, \textit{Surgery on contact 3-manifolds and Stein surfaces}, Bolyai Society Mathematical Studies,Springer-Verlag, Volume 13, 2004.

\bibitem[OzvSz04]{OS:HF}
P. Ozsv{\'a}th and Z. Szab{\'o}, \textit{Holomorphic disks and topological invariants for closed three-manifolds}, Ann. of Math. (2) 159, p.1027–1158, 2004.

\bibitem[OzvSz05]{OS:ContactClass}
P. Ozsv{\'a}th and Z. Szab{\'o}, \textit{Heegaard Floer homology and contact structures}, Duke Math. J. 129, p.39–61, 2005.

\bibitem[P12]{Pardon:HFLecture}
J. Pardon, \textit{Obstruction bundles and counting holomorphic disks in Heegaard Floer homology}, lecture at the Symplectic and Low Dimensional Topologies in Interaction workshop at the Simons Center for Geometry and Physics, \url{http://scgp.stonybrook.edu/video_portal/video.php?id=490}, 2012.

\bibitem[Pard19]{Pardon:Contact}
J. Pardon, \textit{Contact homology and virtual fundamental cycles}, J. Amer. Math. Soc. 32, no. 3, p.825-919, 2019.

\bibitem[Park05]{Parker:Thesis}
B. Parker, \textit{Holomorphic curves in Lagrangian torus fibrations}, Ph.D thesis, Stanford University, 2005.

\bibitem[Pla07]{Olga:CombinatorialCHF}
O. Plamenevskaya, \textit{A combinatorial description of the Heegaard Floer contact invariant},
Algebr. Geom. Topol. 7, p.1201–1209, 2007.

\bibitem[Ra10]{Ras:MilnorConj}
J. Rasmussen, \textit{Khovanov homology and the slice genus}, Invent. math., volume 182, p.419–447, 2010.

\bibitem[Ro90]{Rolfsen}
D. Rolfsen, \textit{Knots and links} (Corrected reprint of the 1976 original), Mathematics Lecture Series, Vol. 7, Publish or Perish Inc., Houston, TX, 1990.


\bibitem[RS93]{RS:Index}
J. Robbin and D. Salamon, \textit{The Maslov index for paths}, Topology 32, Number 4, p.827–844, 1993.


\bibitem[RS01]{RS:Strips}
J. Robbin and D. Salamon, \textit{Asymptotic behavior of holomorphic strips}, Ann. Inst. H. Poincar\'{e} Anal. Non Lin\'{e}aire, p.573–612, 2001.


\bibitem[Ro19]{Rooney:ECH}
J. Rooney, \textit{Cobordism maps in embedded contact homology}, arXiv:1912.01048, 2019.


\bibitem[Ru97]{Rudolph}
L. Rudolph, \textit{The slice genus and the Thurston–Bennequin invariant of a knot}, Proc. Amer. Math. Soc., Volume 125, p.3049–3050, 1997.

\bibitem[Sa99]{Salamon:HamtonianLectures}
D. Salamon, \textit{Lectures on Floer homology}. Symplectic geometry and topology (Park City, UT, 1997), 143–229, IAS/Park City Math. Ser., 7, Amer. Math. Soc., Providence, RI, 1999.


\bibitem[SW10]{SarkarWang}
S. Sarkar and J. Wang \textit{An algorithm for computing some Heegaard Floer homologies}, Ann Math 171, p.1213–1236, 2010.


\bibitem[Sc93]{Schwarz:Morse}
M. Schwarz, \textit{Morse homology}, Progress in Mathematics, vol. 111, Birkh{\"a}user, Basel, 1993

\bibitem[Sc95]{Schwarz:Thesis}
M. Schwarz, \textit{Cohomology operations from $\Circle$ cobordisms in Floer homology}, PhD thesis, ETH Zurich, 1995.


\bibitem[Se07]{BiasedSH}
P. Seidel, \textit{A biased view of symplectic cohomology}, arXiv:math/0704.2055, 2007.

\bibitem[Si19]{Siegel:RSFT}
K. Siegel, \textit{Higher symplectic capacities}, arXiv:1902.01490, 2019

\bibitem[S08]{Siefring:Asymptotics}
R. Siefring, \textit{Relative asymptotic behavior of pseudoholomorphic half-cylinders}, Comm. Pure
Appl. Math. 61, p.1631–1684, 2008.

\bibitem[S13]{Sivek:NoAugs}
S. Sivek, \textit{The contact homology of Legendrian knots with maximal
Thurston-Bennequin invariant}, J. Symp. Geom., Vol. 11, p.167–178, 2013.

\bibitem[S]{Sivek:Code}
S. Sivek, \textit{lch.sage}, software available at \url{https://www.ma.imperial.ac.uk/~ssivek/code/lch.sage}.

\bibitem[Sm62]{Smale:HCobordism}
S. Smale, \textit{On the structure of manifolds}, Amer. Journ. Math., Number 3, p.387-399, 1962.

\bibitem[Su05]{Sullivan:Coproduct}
D. Sullivan, \textit{Sigma models and string topology in Graphs And Patterns}, in Mathematics And Theoretical
Physics: Proceedings Of The Stony Brook Conference On Graphs And Patterns In Mathematics And Theoretical Physics, Dedicated to Dennis Sullivan’s 60th Birthday Proceedings of Symposia in Pure Mathematics,
American Mathematical Society, 2005.

\bibitem[T97]{Traynor:Helix}
L. Traynor, \emph{Legendrian circular helix links}, Mathematical Proceedings of the Cambridge Philosophical Society 122, p.301-314, 1997.


\bibitem[V15]{Vaugon:Bypass}
A. Vaugon, \textit{Reeb periodic orbits after a bypass attachment}, Ergod. Th. Dynam. Sys. 35, p. 615-672, 2015.

\bibitem[V99]{Viterbo:SH}
C. Viterbo, \textit{Functors and computations in Floer homology with applications, Part I}, Geom.
Funct. Anal. 9, p.985–1033, 1999.


\bibitem[W15]{Wand:LegendrianSurgery}
A. Wand, \textit{Tightness is preserved by Legendrian surgery}, Ann. of Math., Volume 128, Issue 2, p.723-738, 2015.


\bibitem[Wa83]{Warner}
F. W. Warner, \textit{Foundations of differentiable manifolds and Lie groups}, Graduate Texts in Mathematics 94, Springer-Verlag, 1983.

\bibitem[Wei91]{Weinstein:Handles}
A. Weinstein, \textit{Contact surgery and symplectic handlebodies}, Hokkaido Math. J., Vol. 20, p.241-251, 1991.

\bibitem[Wen10a]{Wendl:Automatic}
C. Wendl, \textit{Automatic transversality and orbifolds of punctured holomorphic curves in dimension four}, Comment. Math. Helv. 85, no. 2, p.347-407, 2010.

\bibitem[Wen10b]{Wendl:OB}
C. Wendl, \textit{Open book decompositions and stable Hamiltonian structures}, Expo. Math., Vol. 28, p.187-199, 2010.

\bibitem[Wen10c]{Wendl:Foliations}
C. Wendl, \textit{Strongly fillable contact manifolds and J-holomorphic foliations}, Duke Math. J., Volume 151, Number 3, p.337-384, 2010.

\bibitem[Wen10]{Wendl:SHSurvey}
C. Wendl, \textit{A beginners overview of symplectic homology}, Lecture notes available at \url{https://www.mathematik.hu-berlin.de/~wendl/pub/SH.pdf}

\bibitem[Wen13]{Wendl:Torsion}
C. Wendl, \textit{A hierarchy of local symplectic filling obstructions for contact 3-manifolds}, Duke Math. J., Vol. 162, p.2197–2283, 2013.

\bibitem[Wen13]{Wendl:NonExact}
C. Wendl, \textit{Non-exact symplectic cobordisms between contact 3-manifolds}. Journal of Differential Geometry, Volume 95, p.121–182, 2013.

\bibitem[Wen15]{Wendl:Signs}
C. Wendl, \textit{Signs (or how to annoy a symplectic topologist}, blog post available at \url{https://symplecticfieldtheorist.wordpress.com/2015/08/23/signs-or-how-to-annoy-a-symplectic-topologist/}, 2015.


\bibitem[Wen16]{Wendl:SFTNotes}
C. Wendl, \textit{Lectures on Symplectic Field Theory}, arXiv:1612.01009, 2016.

\bibitem[Wi82]{Witten:Morse}
E. Witten, \textit{Supersymmetry and Morse theory}, J. Differential Geom. 17, p.661-692, 1982.

\bibitem[Y06]{Yau:VanishingCH}
M. L. Yau, \emph{Vanishing of the contact homology of overtwisted contact 3-
manifolds (with an appendix by Y. Eliashberg)}, Bull. Inst. Math. Acad. Sin. 1, p.211–229, 2006.
\end{comment}

\end{thebibliography}


\end{document}
